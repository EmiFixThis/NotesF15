
\begin{prob}
How many different ways are there to pick a man and woman who are not married to each other from a group?
\end{prob}


\begin{sol}
Someone $(n)$ and someone else's spouse $(n-1)$ can be picked from a group of $n$ married couples in $n(n-1)$ ways. \textit{(Yes I misread the question before).
\end{sol}


\begin{prob}
How many nonempty words can be formed from three As and five B’s?
(not all letters must be used, any sequence of letters counts as a word)
\end{prob}


\begin{sol}
There are 7 letters available 3 are A's and 4 are B's.
So we have $\frac{7!}{3!4!} = 35$ possible words with repetition. 
\end{sol}


\begin{prob}
How many ternary (0,1,2) sequences of length 10 are there without any two consecutive digits being the same?
\end{prob}

\begin{sol}
For the first slot there is $\dbinom{3}{1}$, the second slot has $\dbinom{2}{1}$, the third also has $\dbinom{2}{1}$ and so on. \\
So we get $3 \times 2^{9}$ different sequences possible. 
\end{sol}


\begin{prob}
How many different outcomes are possible when a pair of dice, one red and one white are rolled two consecutive times?
\end{prob}


\begin{sol}
Each die has 6 sides so it has 6 possible outcomes each time it is rolled. Two dice are rolled so the sample space is $6 \times 6 = 36$.\\

However, each die only has 6 possible outcomes $1, 2, 3, 4, 5, 6$ which is a probability of $\frac{1}{6}$ for any of these numbers to appear. \\

Since each number has the possibility of appearing twice and the dice are independed we multiply (\textit{each roll of the dice constitutes one stage in a multistage process}) to get $\frac{1}{6} \times \frac{1}{6} \times 6 = \frac{1}{6}$.
\end{sol}

\begin{prob}
Construct a perfect cover of an $(8 \times 8)$ chessboard with dominoes $(1 \times 2)$ having no fault-line.

Let $(p \times q) = (8 \times 8)$ and let $(s \times t) = (1 \times 2)$.\\
Assume $pq > st$ and $gcd(s,t)=1$. \\ 
The tiling will include a fault-line if and only if the following conditions hold: 
\end{prob}


\begin{case}
Both $a$ and $b$ divide $p$ or $q$. 
\end{case}
 

\begin{case}
There exist non-zero integers $x,y$ such that $x,y$ can be expressed in at least two ways in the form: $xs+yt$. 
\end{case}
    
    
\begin{case}
If $(s,t) = (1,2)$ then $(p,q) \neq (6,6)$.
\end{case}


\begin{sol}
The first condition is satisfied for both $s$ and $t$.\\
The second condition can be obtained simply by inserting any integer values for $x$ and $y$ and then reversing them. \\
The third condition can be considered a cofactor of the $8 \times 8$ board obtained by removing two rows and two columns. \\
This means that for the given $1 \times 2$ domino, there does not exist any such tiling that will satisfy the last condition. Since the requirement is biconditional the board will always have a fault line if any condition fails.
\end{sol}

\begin{prob}
Show that there is no magic cube of dimension 2.
\end{prob}

\begin{sol}
Assume that a magic square of order three exists. \\

Then for the center cross section the square $k$ must have the value $k = \frac{S}{3}$ where $S$ is the constant sum. \\

Taking the sum of the two diagonals and center column we have: $$(a+k+c)+(d+k+f)+(g+k+h) = 3S = (a,d,g)+(c+f+h)+(3k) $$
This means that $3k = S$. \\

But this is impossible only one center may have this value. \\

$\therefore$ We have a contradiction and there cannot exist a magic cube of order 3 (\textit i.e. Of dimension 2).

\end{sol}