\subsection{Theorems About Primes}


\subsubsection{Chinese Remainder Theorem}


Given $n_{1},n_{2}, \ldots, n_{k}$ are positive integers such that $(n_{i},n_{i+1})$ are relatively prime pairs. 

Then, for any integer sequence $a_{1},a_{2}, \ldots, a_{k}$ we can find an integer $x$ which solves the following system of simultaneous congruences.
$ {\begin{cases}x\equiv a_{1}&{\pmod {n_{1}}}\\\quad \cdots \\x\equiv a_{k}&{\pmod {n_{k}}}\end{cases}} $

\smallskip

Moreover, all $x$ found solutions the system are congruent modulo the product, $N = n_{1} \times n_{2} \times \ldots \times n_{k}$. 

\smallskip

Therefore, 
\[x\equiv y{\pmod {n_{i}}},\quad 1\leq i\leq k\qquad \Longleftrightarrow \qquad x\equiv y{\pmod {N}}. \]

\medskip

Sometimes, the system will have a solution even if the $(n_{i},n_{i+1})$ pairs are not relatively prime. 

We can find a solution $x$ if and only if:
\[ a_{i}\equiv a_{j}{\pmod {\gcd(n_{i},n_{j})}}\qquad \forall i,j \]
All $x$ found solutions will be congruent modulo the least common multiple of the $n_{i}$.



\subsubsection{Euler's Function}

Euler's function $\phi(x)$ is the number of positive integers less than $x$ relatively prime to $x$.

\smallskip

If $\prod_{i=1}^n p^{e_{i}}_{i}$ is the prime factorization of $x$ then
\[\phi(x) = \prod_{i=1}^{n} p^{e_{i} - 1}_{i} (p_{i} - 1)\]

\smallskip

\subsubsection{Euler's theorem}

If $gcg(a,b)=1$ where $a,b \in \mathbb{Z}$ then,
\[ 1 \equiv a^{\phi(b)} \mod{b} \]

\medskip

\subsubsection{Fermat's Little Theorem}
\[1 \equiv a^{p-1} \mod{p} \]

\medskip

\subsubsection{The Euclidean Algorithm}
if $a > b$ and $a,b \in \mathbb{Z}$ where 
\[gcd(a, b) = gcd(a \mod{b}, b).\]


If $\prod_{i=1}^n p^{e_{i}}_i$ is the prime factorization for $x$ then we have
\[S(x) = \sum_{d\vert x} d = \prod_{i=1}^{n} \frac{p^{e_{i}+1}_{i} - 1}{p_{i} - 1}.\]

\subsubsection{Perfect Numbers}
The number $x$ is an even perfect number if and only if $x = 2^{n-1}(2^{n} - 1)$ and $2^{n} - 1$ is prime.


\subsubsection{Wilson's theorem}
The integer $n$ is a prime if and only if:
\[(n-1)! \equiv -1 \mod{n}\]

\pagebreak

\subsubsection{M\"obius inversion:}

$\mu(i) = $
\begin{case}
$1$ if $i = 1$.
\end{case}
\smallskip
\begin{case}
$0$ if $i$ is not square-free.
\end{case}
\smallskip
\begin{case}
$(-1)^{r}$ if $i$ is the product of $r$ distinct primes.
\end{case}


\subsubsection{Sequences}
Given arithmetic sequences $f$ and $g$
\[g(a) = \sum_{d \vert a} f(d)\]
then,

\[f(a) = \sum_{d \vert a} \mu(d) g \bigg( \frac{a}{d} \bigg) \]


\subsubsection{Prime numbers}

\todo{something doesn't look right in here.}

\[p_{n}  = n \ln n + n \ln \ln n - n + n {\ln \ln n \over \ln n}
+ O\bigg({n \over \ln n}\bigg)\]

\[ p_(n) = {n \over \ln n} + {n \over (\ln n)^2} + {2! n \over (\ln n)^3} \\+ O\bigg({n \over (\ln n)^4}\bigg)\]