%%%%%%%%%%%%%%%%%%%%%%%%%%%%%%%%%%%%%%%%%%%%%%%%%%%%%%%%%%%%%%%%%%%%%
% PSTs!!
%%%%%%%%%%%%%%%%%%%%%%%%%%%%%%%%%%%%%%%%%%%%%%%%%%%%%%%%%%%%%%%%%%%%%


\documentclass{article}

%%%%%%%%%%%%%%%%%%%%%%%%%%%%%%%%%%%%%%%%%%%%%%%%%%%%%%%%%%%%%%%%%%%%%
% Package Calls
%%%%%%%%%%%%%%%%%%%%%%%%%%%%%%%%%%%%%%%%%%%%%%%%%%%%%%%%%%%%%%%%%%%%%


% Call fancyhdr package
\usepackage{fancyhdr}
% Call extramarks package
\usepackage{extramarks}
% Call amsmath package
\usepackage{amsmath}
% Call amssymb package
\usepackage{amssymb}
% Call tikz package
\usepackage{tikz}
% Call arev font package
\usepackage{arev}
% Call ifthen package
\usepackage{ifthen}
% Call tcolorbox package
\usepackage{tcolorbox}
% Call fullpage package
\usepackage{fullpage}

%%%%%%%%%%%%%%%%%%%%%%%%%%%%%%%%%%%%%%%%%%%%%%%%%%%%%%%%%%%%%%%%%%%%%



%%%%%%%%%%%%%%%%%%%%%%%%%%%%%%%%%%%%%%%%%%%%%%%%%%%%%%%%%%%%%%%%%%%%%
% Begin List of PSTs
\begin{document}


\begin{tcolorbox}
    \textbf{\large List of Definitions for Math304}
List of Definitions, Theorems, and other mathematical statements for Math304. \\
\end{tcolorbox}
\bigskip


%%%%%%%%%%%%%%%%%%%%%%%%%%%%%%%%%%%%%%%%%%%%%%%%%%%%%%%%%%%%%%%%%%%%%
% Begin Definitions

\section*{Four Basic Counting Rules}



\begin{tcolorbox}
\textbf{Partition} \\
Let $\mathcal{S}$ be a set. 
\\
A \textit{partition} of $\mathcal{S}$ is a collection $\mathcal{S}_{1},\mathcal{S}_{2}, …, \mathcal{S}_{m}$ of subsets of $\mathcal{S}$ such that every element of $\mathcal{S}$ is in exactly one of the subsets:
\\ \\

$$\mathcal{S} = \mathcal{S}_{1} \cup \mathcal{S}_{2} \cup … \cup \mathcal{S}_{m},$$ \\
$$\mathcal{S}_{i} \cap \mathcal{S}_{j} = \emptyset, (i \neq j).$$


Therefore, the sets $\mathcal{S}_1, \mathcal{S}_2, …, \mathcal{S}_m$ are pairwise disjoint sets, which as a union form $\mathcal{S}$. 
\\
The subsets $\mathcal{S}_1, \mathcal{S}_2, …, \mathcal{S}_m$ are called the \textit{parts} of the partion. 
\\
\textbf{Note: } By this definition is possible that some part $\mathcal{S}_{i} = \emptyset$. \\
If so there is no advantage to this fact. 
\\
\textbf{Consider: } Partitions with one or more empty parts. 
\\ 
Then the number of objects in the set $\mathcal{S}$ is denoted |$\mathcal{S}$| and is called the \textit{size} or \textit{magnitude} of $\mathcal{S}$. 
\\ \\
\end{tcolorbox}



%%%%%%%%%%%%%%%%%%%%%%%%%%%%%%%%%%%%%%%%%%%%%%%%%%%%%%%%%%%%%%%%%%%%%
% End Document
\end{document}	