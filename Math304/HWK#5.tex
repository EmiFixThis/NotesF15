\documentclass{article}

\usepackage{amsmath,amssymb,amsthm}

\begin{document}

\section*{Problem}
  "In any sequence $a_{i},a_{2},\ldots,a_{mn+1}$ of $mn+1$ distinct real numbers, there exists an increasing subsequence $$ $a_{i_1}$ < a_{i_2} < \ldots < a_{i_m+1}$$ for $$(i_{1}<i_{2}< \ldots <i_{m+1})$$ of length $m+1$ \\ \\
  $$a_{j_1} < a_{j_2} < \ldots < a_{j_n+1}$$ for $$(j_{1}<j_{2}< \ldots <j_{n+1})$$ 
  of length $n+1$ or both." 
  \bigskip

\section*{Problem}
    \textbf{Prove that in a group of $n > 1$ people there are two who have the same number of acquaintances in the group. (It is assumed that no one is acquainted with oneself.)}
    \bigskip
    
    \textbf{Solution:} \\
        \bigskip
        
        Let $G$ be the group and $n = |G|$, such that $n > 1$.
        \\ \\
        \textbf{Test Case($n=2$)}
        
        \bigskip
        
        Let $n_{1}, n_{2} \in G$ where $|G|=2$; then we have that $n_{1}$ knows $n_{2}$, likewise $n_{2}$ knows $n_{1}$. 
        
        \newline
        
        Then \textit{"I know you and you know me"} so the base case is true since everyone knows one person.
        
        \newline
        
        Conversely, if $n=2$ and neither $n_{1}$ nor $n_{2}$ know one another the base case is true, since both people know zero people in the group.
        
        \newline
        
        \textbf{Count the Complement:}
        
        \newline
        
        By supposition $n > 1$, and by assumption no one in the group can be aquainted with themselves. So for some $n_{i} \in G$ there must be some $n_{j} \in G$ that knows the same number of people as $n_{i}$; if there is not anyone in the group who doesn't know anyone else.  
        
        \newline
        
        If there is at least one (or $k$) person (people) who don't know anyone else, we count that as the complement of people who do know someone and we have the same result.
         \newline
         
         we just reduce the number of people who do know at least one other person in the group by the one (or $k$) number of people do not know anyone in the group.
        
        \newline
        
        These cases are independent, and complements in $G$, so showing either case (counting the complement), shows the other as well. 
        
        \medskip
        
        $\blacksquare$
        
        \medskip
        

    \bigskip
    
\section*{Problem}
    \textbf{Combinatorially prove the following binomial identity
$$\sum_{k=0}^{r} {n+k\choose k} = { n + r + 1 \choose r }.$$}
    
    \bigskip

    \textbf{Proof}(\textit{Combinatorial}):
    
    \newline
    
    Let $S$ be a set of positive integers.
    Then $\mathcal{P}(S)$ denotes the power set of $S$, defined as the set of all subsets of $S$, including the empty set and $S$.
    \newline
    If $|S| = n$, the $|\mathcal{P}(S)|=2^{n}$, since for every subset the question is a binary (Boolean) type of question the membership in the subset of any element of $S$.
    \newline
    Namely, for elements $x,y \in \mathcal{P}(S) : \mathcal{P}(S) = \{(x_{1},y_{1}),(x_{2},y_{2}), \hdots,(x_{n},y_{n})\}$, for some ordered pair $(x_{i},y_{i})$ we ask if it is in some $M \subset \mathcal{P}(S)$. 
    \newline
    The answer of this membership can only ever be Yes or No for any element in $\mathcal{P}(S)$, the question is also binary for any position in $\mathcal{P}(S)$.
    \newline
    So we are essentially asking how many ways can we arrange pairs of elements of $S$, if we don't care about repetition.
    \newline
    
    
%%%%%%%%%%%%%%%%%%%%%%%%%%%%%%%%%%%%%%%%%%%%%%%%%%%%%%%%%%%%%%%%%%%%%
% End Document
\end{document}