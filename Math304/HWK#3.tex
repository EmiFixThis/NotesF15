\documentclass{article}

\usepackage{amsmath,amssymb,amsthm}

\begin{document}
 
\section*{Problem 1}
 How many sets of three integers between 1 and 20 are possible if no two consecutive integers are to be in a set?
  \bigskip
  
  \subsection*{Solution}
      There are $\mathcal{P}(20,3) = 1140$ number of combinations of 3-element subsets of numbers from 1 to 20. 
      \\ \\
      If three integers are consecutive they have the form: \\
      $a_{0}=1,a_{1}=2,a_{2}=3$ \\
      $a_{n},a+{n+1},a_{n+2}$. 
      \\ \\
      We want to form subsets that dont have two consecutive integers.
      \\ \\ 
      So choose $a_{n+1}$ to not be any of the subsets. \\
      Then $a_{n}$ and $a_{n+2}$ are in one set, and $a_{n+1}$ is in the other. 
      \\ \\
      If we line up the 20 integers in increasing order we have that either both $a_{n}$ and $a_{n+2}$ will be odd, or both will be even, and $a_{n+1}$ will have opposite parity. 
      \\ \\
      So to get from some odd number $2i+1$ to the next odd number $2i+3$ we just subtract the even number between them. Likewise to get from one even number to the next we subtract the odd number between them. 
      \\ \\
      Between any two even (or odd) numbers we will have two dividers:
      \\ \\
      1  2  3  4
      \\ \\
      1  2 | 3 | 4 
      So if we want subsets of size $k$ There will be $k-1$ dividers, with two between each number so we have $2(k-1)$ dividers which can be placed in $k+1$ spaces between even (or odd) numbers. 
      \\ \\
      20 numbers in $k$ subsets and $2(k-1)$ dividers for $k+1$ spaces:
      $20-k-2(k-1) = 20 - k -2k+2 = 20 - 3k + 2$
      \\ 
      Then $20-3k+2 + (k+1) - 1 = 20-2k+2$ when $k=3$ we have 16 $\binom{16}{3} = 560$
  
\section*{Problem 2}
    There are $n\geq4$ knights seated around King Arthur's round table and three of them are selected to be sent off to slay a troublesome dragon. What is the probability that at least two of the three were seated next to each other?
    \bigskip
    
    \textbf{Solution:}
    \bigskip
    
    Suppose that $n \geq 4$ and we select three at random to go kill some poor dragon, (who is probably just misunderstood maybe it ate some sheep or something dragon's probably don't have much of a concept of ownership and they undoubtedly have to eat). 
    \\ \\
    \textbf{Count the Complement}
    \\ 
    Suppose we don't want the knights chosen to be knights which were seated next to each other.
    \\ \\
    The total number of ways this can happen is $C = \binom{n-3}{3}$; to fix the overcounting due to rotations divide by $(n-3)$ to get $C/(n-3)$
    \\ \\
    Let $T$ be the number of total ways to choose 3 from n this is $T = \binom{n}{3}$; adjust for the rotations again by dividing by n to get $T/n$.
    \\ \\
    Now subtract $C/(n-3)$ from the adjusted total and divide by the adjusted total (because we want a probability) to get:
    $$\frac{\frac{\binom{n}{3}}{n} - \frac{\binom{n-3}{3}}{(n-3)}}{\frac{\binom{n}{3}}{n}}= \frac{6(n-3)}{(n-2)(n-1)}$$
    \bigskip
    \textit{not to sure about how I simplified the right hand side.}
 
\section*{Problem 3}

