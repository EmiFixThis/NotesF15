\documentclass{article}

\usepackage{amsmath, amssymb, amsthm}


\begin{document}
 
\section*{Problem 1: }
How many different ways are there to pick a man and woman who are not married to each other from a group of n married couples? 
\\ \\

\subsubsection*{Warm Up: Non-traditional Marriages} 
To warm up consider the case for $n = 10$ where the order doesn't matter and no one has gotten married yet (\textit{because it's more fun this way, so I'll remember it maybe more}).
\\ \\

\textbf{WLOG:} Line the boys and girls up in one big line then randomly choose someone to be the first and pick someone for them to marry. 
\\ \\
After you have chosen a person there are 9 people you can marry them off to, then you send them away (\textit{they do not get back in line. Perhaps they went on a honeymoon; perhaps they are in a room somewhere awkwardly staring at each other in shock}). \\

Now you randomly pick another person (person 8) and choose someone for them to marry, you have 7 people to choose from. \\ \\
So we have: $$\mathcal{P}(n,2k-1) = (2k-1)(2k-2)...(1) = 9\cdot7\cdot5\cdot ... \cdot 3 \cdot1 = 945$$
\\ \\

\subsubsection*{TestCase: Finite Traditional Marriage} 
Again choose $n=10$, but this time make it traditional marriages (european traditional if you will boy-girl pairs). \\
Line all the boys up in one row and the girls in another facing one another.
Order the lines by some fixed arbitrary specification, say by height largest to smallest (it doesn't really matter, we can even skip this all together). \\
Arbitrarly choose someone and then choose their opposite sex mate. 
\\ \\
For step zero we have 10 scared unmarried boys and girls now in \textbf{two} lines. \\
Choose the first person in one of the lines and marry them off to the first person in the opposite line. \\
Pairing them off and sending them away gives: $(5-1)(5-2)...(5-4) = 5!$, but there were 10 people all together. So this is really 5 pairs of two or $5!$. 
\\ \\ 

\subsection*{Actual Solution: }
Someone $(n)$ and someone else's spouse $(n-1)$ can be picked from a group of $n$ married couples in $n(n-1)$ ways. \textit{(Yes I misread the question before).} 
\\ \\

\section*{Problem 2: }
How many nonempty words can be formed from three As and five B’s?
\textit{(not all letters must be used, any sequence of letters counts as a word)}
\\ \\
There are 7 letters available 3 are A's and 4 are B's.
So we have $\frac{7!}{3!4!} = 35$ possible words with repetition. 
\\ \\
\section*{Problem 3: }
\textbf{How many ternary (0,1,2) sequences of length 10 are there without any two consecutive digits being the same?}
\\ \\ 
\subsection*{Solution: }
For the first slot there is $\dbinom{3}{1}$, the second slot has $\dbinom{2}{1}$, the third also has $\dbinom{2}{1}$ and so on. \\
So we get $3 \times 2^{9}$ different sequences possible. 
\\ \\
\section*{Problem 4: }
How many different outcomes are possible when a pair of dice, one red and one white are rolled two consecutive times?
\\ \\
\subsection*{Solution: }
Each die has 6 sides so it has 6 possible outcomes each time it is rolled. Two dice are rolled so the sample space is $6 \times 6 = 36$.\\
However, each die only has 6 possible outcomes $1, 2, 3, 4, 5, 6$ which is a probability of $\frac{1}{6}$ for any of these numbers to appear. \\
Since each number has the possibility of appearing twice and the dice are independed we multiply (\textit{each roll of the dice constitutes one stage in a multistage process}) to get $\frac{1}{6} \times \frac{1}{6} \times 6 = \frac{1}{6}$.
\\ \\
\section*{Problem 5: }
Construct a perfect cover of an $(8 \times 8)$ chessboard with dominoes $(1 \times 2)$ having no fault-line.
\\ \\
Let $(p \times q) = (8 \times 8)$ and let $(s \times t) = (1 \times 2)$.\\
Assume $pq > st$ and $gcd(s,t)=1$. \\ 
The tiling will include a fault-line if and only if the following conditions hold: 
\begin{itemize}
    \item Both $a$ and $b$ divide $p$ or $q$. 
    \item There exist non-zero integers $x,y$ such that $x,y$ can be expressed in at least two ways in the form: $xs+yt$. 
    \item If $(s,t) = (1,2)$ then $(p,q) \neq (6,6)$.
\end{itemize}

\subsection*{Solution: }
The first condition is satisfied for both $s$ and $t$.\\
The second condition can be obtained simply by inserting any integer values for $x$ and $y$ and then reversing them. \\
The third condition can be considered a cofactor of the $8 \times 8$ board obtained by removing two rows and two columns. \\
This means that for the given $1 \times 2$ domino, there does not exist any such tiling that will satisfy the last condition. Since the requirement is biconditional the board will always have a fault line if any condition fails. \\ \\

\section*{Problem 6: }
\textbf{Show that there is no magic cube of dimension 2.}\\ \\
\subsubsection*{Proof: }
Assume that a magic square of order three exists. \\
Then for the center cross section the square $k$ must have the value $k = \frac{S}{3}$ where $S$ is the constant sum. \\
Taking the sum of the two diagonals and center column we have: $$(a+k+c)+(d+k+f)+(g+k+h) = 3S = (a,d,g)+(c+f+h)+(3k) $$
This means that $3k = S$. \\
But this is impossible only one center may have this value. \\
$\therefore$ Contradiction. \\
There cannot exist a magic cube of order 3 (\textit i.e. Of dimension 2).



\end{document}