%%%%%%%%%%%%%%%%%%%%%%%%%%%%%%%%%%%%%%%%%%%%%%%%%%%%%%%%%%%%%%%%%%%%%
% PSTs!!
%%%%%%%%%%%%%%%%%%%%%%%%%%%%%%%%%%%%%%%%%%%%%%%%%%%%%%%%%%%%%%%%%%%%%


\documentclass{article}

%%%%%%%%%%%%%%%%%%%%%%%%%%%%%%%%%%%%%%%%%%%%%%%%%%%%%%%%%%%%%%%%%%%%%
% Package Calls
%%%%%%%%%%%%%%%%%%%%%%%%%%%%%%%%%%%%%%%%%%%%%%%%%%%%%%%%%%%%%%%%%%%%%


% Call fancyhdr package
\usepackage{fancyhdr}
% Call extramarks package
\usepackage{extramarks}
% Call amsmath package
\usepackage{amsmath}
% Call amssymb package
\usepackage{amssymb}
% Call tikz package
\usepackage{tikz}
% Call arev font package
\usepackage{arev}
% Call ifthen package
\usepackage{ifthen}
% Call tcolorbox package
\usepackage{tcolorbox}
% Call fullpage package
\usepackage{fullpage}

%%%%%%%%%%%%%%%%%%%%%%%%%%%%%%%%%%%%%%%%%%%%%%%%%%%%%%%%%%%%%%%%%%%%%

% PST Construction
\newtheorem{pst}{PST}

% tcolorbox specification
\newtcolorbox{mybox}[1]{minipage boxed title, 
    enhanced,attach boxed title to top center= 
        {yshift=-3mm,yshifttext=-1mm}, 
    boxed title style={size=small,colback=red}, 
    center title,title={#1}}
%\begin{mybox}{My title} 
%    \lipsum[2] 
%\end{mybox}

%%%%%%%%%%%%%%%%%%%%%%%%%%%%%%%%%%%%%%%%%%%%%%%%%%%%%%%%%%%%%%%%%%%%%
% Begin List of PSTs
\begin{document}


\begin{mybox}{List of PSTs!}
This list of PSTs are collected from \textit{"A Decade of the Berkeley Math Circle" by Zvezdelina Stankova and Tom Rike.}\\
\begin{comment}
    \textit{Zveda if you ever see this I am sorry, you were my very favorite and taught me more than you know.}
\end{comment}
\end{mybox}\bigskip


%%%%%%%%%%%%%%%%%%%%%%%%%%%%%%%%%%%%%%%%%%%%%%%%%%%%%%%%%%%%%%%%%%%%%
% Begin PSTs

\section*{Combinatorics}


\subsection*{Menus Make You Multiply}

\begin{mybox}{\pst \textbf{Menus}}
If the thing we are counting is the outcome of a \textit{multi-stage process}, then the number of the outcomes \textit{the product of the number of choices for each stage}. 
\end{mybox}
\\ \\


\begin{mybox}{\textbf{Encoding}}
When creating and using an encoding method, you must verify that there is a \textit{one-to-one correspondence} between the set of encodings and all possible outcomes in the problem. \\
i.e. That every encoding corresponds to exactly one possible outcome, and every possible outcome corresponds to exactly one encoding. 
\end{mybox}
\\ \\

\subsubsection*{\large Limitations of Multiplication: \textit{\small Multiplication works as long as the number of \\ decisions required at each stage stays constant.}}\bigskip


\begin{mybox}{\textbf{Keep the Chaos Down}}
To keep the chaos down when counting and to have some control over what is happening, it is helpful to \textit{order all of the involved object/people (or some subset of them)}. This order can be random or according to some specific criterion. 
\end{mybox}
\\ \\

\begin{mybox}{\textbf{Mutually Excusive Cases}}
Break the outcomes into several separate, \textit{mutually exclusive cases}. \\
i.e. Every possible outcome must belong to exactly one of these cases. 
\end{mybox}
\\ \\


\subsection*{Partition Leads to Addition}
\bigskip


\begin{mybox}{\textbf{Count the Total Outcomes}}
Whenever we partition the outcomes of something into several cases, each requiring different counting methods, we \textit{add the number of outcomes} in each case to get the total number of outcomes. 
\end{mybox}
\\ \\


\subsection*{Counting the Complement}
\bigskip


\begin{mybox}{I Like This But I Don't Like That}
    The method of \textit{counting the complement}, is used when we can partition the total set of outcomes into the things we are interested in, and the resti (it's \textit{negation or complement}). \\
    If the total is easy to count and the negation is easy to count, then we\\ \textit{count the complement} and our answer is just the difference of the two.  
\end{mybox}
\\ \\


%%%%%%%%%%%%%%%%%%%%%%%%%%%%%%%%%%%%%%%%%%%%%%%%%%%%%%%%%%%%%%%%%%%%%
% End Document
\end{document}	